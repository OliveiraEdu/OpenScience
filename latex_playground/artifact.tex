\documentclass{article}
\usepackage{longtable}
\usepackage{multirow}
\usepackage{array}
\usepackage{graphicx}  % For inserting images
\usepackage{caption}   % For better caption formatting
\usepackage{float}     % To control figure placement


\title{Open Science Platform Artifact}
\author{Eduardo Oliveira}
\date{\today}

\begin{document}

\maketitle

\section{Open Science Platform}

\subsection{Overview}

The Open Science platform aims to empower researchers and members of the scientific community by providing a secure, transparent, traceable and tamper-proof environment for sharing project artifacts and data. Building on this objective, the platform leverages decentralized technologies to ensure the integrity and reliability of shared information.


\subsection{Technology Stack}

The Open Science platform is built upon a robust technical foundation, comprising:

\begin{itemize}
    \item Hyperledger Iroha v1 Blockchain: The core infrastructure for account management and transaction recording and business rules enforcement through Smart Contracts ensuring secure and transparent data exchange.
    \item IPFS (InterPlanetary File System): The decentralized storage for project artifacts and metadata, guaranteeing tamper-proof and persistent access to shared information.
\end{itemize}


Aside from the decentralized technologies above, the platform also relies on the following off-chain, centralized components:

\begin{itemize}
    \item Jupyter Notebooks in Python: The front-end interface of the platform leverages Jupyter Notebooks in Python to automate and display the execution steps of the activities in the platform.
    \item Apache Tika: Utilized for extracting file metadata, enhancing the platform's ability to manage and describe artifact content.
    \item Woosh: For efficient indexing and search capabilities for artifacts stored on the platform.
\end{itemize}

\subsection{Operations}

The Open Science platform is comprised of the following operations:

User enrollment



\subsection{Benefits}

The Open Science platform offers numerous benefits for researchers and members of the scientific community, including:

\begin{itemize}
    \item Secure data sharing: By utilizing blockchain technology and IPFS, the platform ensures tamper-proof data exchange.
    \item Transparent data management: The use of smart contracts and decentralized storage guarantees transparency in data access and modification history.
    \item Collaborative research environment: The platform enables researchers to collaborate on projects, share artifacts and results, and track progress.
\end{itemize}

\subsection{Challenges}

The Open Science platform faces several challenges, including:

\begin{itemize}
    \item Scalability: As the number of users increases, the platform needs to be able to handle a growing amount of data and transactions efficiently.
    \item Interoperability: Ensuring seamless integration with existing research platforms and tools is crucial for widespread adoption.
    \item User Adoption: Educating researchers about the benefits of decentralized technologies and the Open Science platform can be an uphill battle.
\end{itemize}

\subsection{Future Work}

The Open Science platform has several areas for future development, including:

\begin{itemize}
    \item Integration with existing research platforms: Collaborations with established research platforms to expand the platform's reach and user base.
    \item Enhanced security measures: Implementing additional security protocols to protect against potential threats and maintain the integrity of shared information.
    \item User interface improvements: Enhancing the web interface to make it more user-friendly and accessible for researchers from diverse backgrounds.
\end{itemize}

\section{Conclusion}
The Open Science platform is a comprehensive solution for secure, transparent, traceable, and tamper-proof data sharing and collaboration. By leveraging decentralized technologies, the platform empowers researchers to share project artifacts and data in a reliable and trustworthy manner.

\end{document}



