User Accounts Table

\begin{table}[h]
    \centering
    \begin{tabular}{|l|l|}
        \hline
        \textbf{Attribute}    & \textbf{Description} \
        \hline
        foaf:name             & Researcher's full name \
        foaf:mbox             & Researcher's email \
        foaf:organization     & Affiliated institution \
        schema:identifier     & Unique identifier (e.g., ORCID) \
        foaf:holdsAccount     & Online account details (ID, role, public key) \
        schema:linked_project & Associated research project ID \
        \hline
    \end{tabular}
    \caption{User Account Attributes}
    \label{tab:user_accounts}
\end{table}

Research Projects Table

\begin{table}[h]
    \centering
    \begin{tabular}{|l|l|}
        \hline
        \textbf{Attribute} & \textbf{Description} \
        \hline
        schema:identifier  & Unique project identifier \
        schema:publicKey   & Cryptographic key \
        schema:name        & Project name \
        dc:abstract        & Research abstract \
        schema:keywords    & Keywords describing the project \
        schema:startDate   & Start date of the project \
        schema:endDate     & End date of the project \
        schema:funding     & Funding organization details \
        schema:location    & Project location \
        schema:metadataCID & IPFS metadata content ID \
        schema:linked_user & Associated researcher ID \
        \hline
    \end{tabular}
    \caption{Research Project Attributes}
    \label{tab:research_projects}
\end{table}




\section{Introduction_old}


The reproducibility crisis in science arises when researchers fail to replicate the results of a study, even when using the same methods and materials. This issue is prevalent across scientific disciplines where replication is essential for validating findings, advancing knowledge and maintaining public trust in research.

Concerns over transparency and reproducibility have intensified in recent years. Studies indicate that a significant proportion of published research does not withstand rigorous scrutiny when retested, raising doubts about the reliability of scientific knowledge. This crisis results in wasted resources, misdirected efforts and diminished confidence in research findings. Despite various efforts to improve reproducibility, such as better reporting standards and Open Science initiatives, significant challenges remain in ensuring that research findings are consistent, transparent and verifiable.

Several factors contribute to this challenge, including flawed study designs, data quality issues, unreliable measurement tools and inconsistent research practices. Poorly designed studies may overlook critical variables, while inadequate measurement tools can produce misleading results. Furthermore, incomplete reporting or the omission of key methodological details can hinder independent verification of findings. Even existing Open Science practices, while valuable, face limitations in scalability, consistency and automation, often requiring extensive manual effort to achieve reproducibility.

A promising approach to addressing these issues is the adoption of decentralized technologies that enhance transparency, accessibility and collaboration. Open Science advocates for the sharing of research outputs such as publications, datasets, software and methodologies to enable verification and reuse. However, existing systems and tools are often fragmented, not fully transparent  or prone to manipulation. Decentralized technologies such as blockchain, IPFS and smart contracts offer innovative solutions to these problems by providing secure, transparent and automated means of data sharing, validation and record-keeping. These technologies are ideally suited for enabling reproducibility by ensuring the immutability of records, enhancing trust in data and promoting more consistent sharing practices across diverse research fields.

To address the challenges of reproducibility, this study proposes a structured artifact designed to provide a transparent framework for documenting, validating and disseminating research outputs. This artifact will support a wide range of materials, including reports, protocols, datasets, images and videos. By ensuring secure and accessible record-keeping, the artifact aims to enhance methodological rigor and enable independent verification of scientific findings.





The core focus will be on the following decentralized technologies and their potential applications in the realm of reproducibility in scientific research:

\begin{itemize}
    \item \textbf{Blockchain}: A distributed, immutable ledger technology that can securely record and verify transactions across a peer-to-peer network. Blockchain's transparency, immutability, and security can be harnessed to track data provenance, ensure data integrity, and validate research processes \cite{christidis2016blockchain}.

    \item \textbf{InterPlanetary File System (IPFS)}: A peer-to-peer distributed file storage system that uses content-addressed, peer-distributed files and networking protocols to create a permanent, decentralized network. IPFS can facilitate secure, tamper-evident data sharing and hosting, essential for preserving and reproducing research outputs \cite{benet2014ipfs}.

    \item \textbf{Smart Contracts}: Self-executing contracts with the terms of the agreement directly written into code. Smart contracts can automate tasks, enforce agreements, and enforce data input, thus streamlining research processes and ensuring methodological rigor \cite{buterin2014ethereum}.
\end{itemize}

This dissertation will investigate the underlying principles and functionalities of these technologies, followed by an in-depth analysis of their potential to enhance reproducibility in scientific research. The practical implementation and evaluation of these technologies in a real-world scientific research scenario will also be explored to demonstrate their feasibility and assess their tangible benefits.


\subsection{Blockchain and Open Science}

The relation between reproducibility and Open Science is crucial, as it ensures that findings can be duplicated and verified without artificial constraints or obstacles. This allows researchers to build upon each other's work, fostering a collaborative environment where knowledge is shared freely for everyone to benefit. By promoting transparency in research, the scientific community aims to reduce the risk of misconduct, increase accountability, and ensure that scientific progress can be sustained over time without setbacks due to irreproducible findings or biased data manipulation.

Examples of Open Science applications include:
\begin{itemize}
    \item Making research articles openly accessible.
    \item Encouraging the use of open-source software in scientific computing and data analysis.
    \item Promoting transparent communication and reporting of methods and results.
    \item Ensuring that all data used for experiments or analyses is made available to others.
    \item Providing detailed documentation about experimental procedures.
    \item Following best practices like preregistration of studies before data collection.
\end{itemize}

By implementing these strategies, researchers can ensure their work is more accessible and usable by the broader scientific community, allowing for collaboration and building upon each other's discoveries without artificial constraints. This ultimately contributes to advancing the understanding and development of science in a more efficient and sustainable manner.

\subsection{Technical Approaches: Blockchain, Smart Contracts, and IPFS}

The use of blockchain, smart contracts, and IPFS can significantly enhance the reproducibility of scientific research through Open Science. Blockchain technology allows for secure, transparent, and tamper-proof record-keeping and data storage. Smart contracts ensure that complex rules are executed automatically when certain conditions are met, which can streamline processes such as peer review and submission of research articles. By utilizing these technologies, researchers can maintain a permanent and auditable record of their work, providing an additional layer of transparency and accountability.

IPFS enables the sharing and interconnection of large files across the network, facilitating collaboration among researchers working on complex projects. It also allows for decentralized data storage, reducing reliance on centralized servers and minimizing the risk of data loss due to technical failures or intentional sabotage. The combination of these technologies can create a robust infrastructure that supports the open-source development and sharing of scientific knowledge, ultimately enhancing research reproducibility and collaboration among scientists worldwide.

\subsection{Potential Applications of a Decentralized Open Science Platform}

A platform encompassing the aforementioned technologies could be applied in various scientific use cases:

\begin{itemize}
    \item \textbf{Decentralized Repositories}: Researchers can store and share research findings in an open-access manner with clear guidelines for transparency and reproducibility.
    \item \textbf{Smart Contracts for Peer Review}: Automated peer review processes ensuring thorough manuscript vetting before publication.
    \item \textbf{Decentralized Collaboration}: IPFS can enable researchers to work together on large-scale projects while reducing reliance on centralized data storage.
\end{itemize}

In summary, the combination of blockchain, smart contracts, and IPFS has vast potential for improving research reproducibility through Open Science. By providing secure, transparent, and auditable records of research activities, these technologies can help reduce corruption in scientific research, increase accountability among researchers, and foster collaboration between scientists worldwide.
