Preclinical research is pivotal for advancing life sciences, aiding in the discovery of pathological mechanisms and fostering the development of therapeutic strategies, new drugs and vaccines. However, the reliability and credibility of research findings have come under scrutiny due to significant challenges related to reproducibility. Reproducibility, the ability of researchers to independently replicate and verify study results, is fundamental to scientific integrity and progress. A staggering 51\% to 89\% of preclinical studies are irreproducible, leading to waste in human and material resources and a deceleration of scientific progress ~\cite{freedman_economics_2015}.

Factors leading to irreproducible research on preclinical studies fall into four major categories ~\cite{freedman_economics_2015}: The first category (C1) represents biological reagents and reference materials, the second category (C2) is related to study design, the third category (C3) to data analysis and reporting and the fourth category (C4) to laboratory protocols.

Biological reagents and reference materials concentrates 36.1\% of the failures in reproducing experiments, usually related to issues with non validated or contaminated reagents and inadequacies in biological materials handling. 

Study design presents 27.6\% of the factors leading to irreproducibility, it is largely associated to inconsistencies or lack of methods for sampling, randomisation and blinding. 

Data analysis and reporting concentrates 25.5\% of the total factors and is linked to poor statistical methods, unclear criteria for missing data, data inclusion and exclusion and outliers handling. Topics such as absence of disclosure of full results and primary data also constitutes factors in this category. 

Laboratory protocols accumulates 10.8\% of the overall causes and correlates to lab operations not adhering to standards and best practices when conducting preclinical experiments.

Traditionally, research data, the foundation of scientific investigation, suffers from challenges like tampering, selective reporting and restricted accessibility. This creates a barrier to transparency, hindering verification and impeding the construction of a robust body of knowledge.

The reproducibility crisis impacts nearly every field of scientific research and requires a radical shift, to address the fundamental issues of data integrity, opacity and inefficient research practices.

To address these challenges, the Open Science movement champions the core tenets of transparency, collaboration and accessibility in scientific research. By advocating for the open sharing of data, methods and findings, Open Science seeks to enhance reproducibility and ensure the reliability of scientific outcomes. However, the current Open Science ecosystem encounters limitations in data integrity, accessibility and effective governance, hindering its ability to fully resolve the reproducibility crisis.

Emergent technologies such as blockchain, the InterPlanetary File System (IPFS) and smart contracts offer a promising solution to these issues. Blockchain technology, with its core predicates of immutability, resilience, traceability, distribution and programmability, can create an unalterable record of research processes and data. When combined with IPFS for long-term data preservation and accessibility and smart contracts for automating and enforcing transparent governance in scientific collaborations, these technologies can together enable a decentralized application (DApp) specifically designed to enforce reproducibility in scientific research.

A decentralized platform is essential for effectively leveraging these technologies. Such a model empowers researchers to collaboratively establish and evolve standards, data formats and validation protocols, fostering a community-driven ecosystem that upholds methodological transparency and data integrity. This approach aligns with the principles of Open Science, promoting openness, inclusivity and accountability.

This dissertation explores how blockchain, DApps, IPFS and smart contracts, as the building blocks of a DApp, can enhance reproducibility in scientific research through a innovative platform. By investigating the technical foundations, challenges and practical implementations of this innovative approach, the study aims to contribute to advancing scientific research practices and restoring confidence in research findings.

In the following sections, the dissertation provides a comprehensive review of relevant literature, presents the proposed decentralized model, validates its effectiveness according to the technical predicates and discusses implications and future directions. Through this exploration, the dissertation strives to reshape how scientific research is conducted, shared and validated in the era of Open Science. This is not just about ensuring the validity of scientific knowledge, it is about fostering a culture where scientific experiments are not only validated but also readily accessible to all.
