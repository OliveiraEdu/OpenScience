\begin{document}

The reproducibility crisis in scientific research refers to the inability to replicate or reproduce results from previous studies, leading to questions about the reliability and trustworthiness of published findings. This issue can be attributed to various factors, including poor study design, flawed measurement tools, inadequate reporting of methods and data, and a lack of transparency in research practices ~\cite{freedman_economics_2015}.

Studies have shown that a significant number of studies fail to hold up under rigorous scrutiny when retested or replicated, with estimates suggesting that only 36\% of preclinical medical research is reproducible ~\cite{begley_reproducibility_2015}. This crisis has serious consequences for the scientific community and society as a whole, including wasted resources, misleading policy decisions, and delayed progress in fields like medicine and science.

Open Science is an approach to conducting and disseminating research that emphasizes transparency, collaboration, and reproducibility. The core principles of Open Science include making raw data, methods, and materials publicly available; using open-source software and hardware; and facilitating community involvement and feedback (Pain, 2017). By adopting these practices, researchers can increase the reliability and trustworthiness of their findings, improve collaboration and knowledge sharing, and accelerate progress in various fields.

Open Science could potentially address the reproducibility crisis by providing a framework for transparency and accountability in research practices. By making raw data and methods publicly available, researchers can facilitate the replication and verification of results, reducing errors and biases. Additionally, Open Science encourages collaboration and community involvement, which can help identify flaws and areas for improvement in study design, methodology, and reporting.

The Open Science Framework (OSF) ~\cite{foster_open_2017} is an innovative tool that has revolutionized the way researchers collaborate, share, and preserve scientific data. Its benefits are undeniable, providing a centralized hub for project management, collaboration, and reproducibility. The OSF's user-friendly interface and feature-rich environment have made it an invaluable resource for scientists worldwide. However, the OSF framework poses important limitations due to its centralized approach. In contrast to the OSF, a platform incorporating decentralized technologies such as blockchain, smart contracts, and IPFS could offer a more flexible, transparent, and secure framework for Open Science. By leveraging the distributed nature of these technologies, we can create a system where data is stored in a tamper-proof and censorship-resistant manner, ensuring the integrity and provenance of scientific findings. Additionally, decentralized storage solutions like IPFS can enable researchers to store and share large datasets and other scientific artifacts as well without relying on centralized servers, reducing the risk of data loss or unauthorized access. Ultimately, a platform that combines the strengths of both approaches could provide an unparalleled level of transparency, reproducibility, and collaboration in Open Science.

The technical support of blockchain technology, smart contracts, and IPFS (InterPlanetary File System) can further enhance the adoption of Open Science principles (Bollen et al., 2018). Blockchain's decentralized, secure, and transparent nature allows for the creation of immutable records of research data, methods, and findings. Smart contracts can automate peer review processes and manuscript vetting, ensuring that published articles meet rigorous standards. IPFS enables the decentralized storage and sharing of large datasets and other scientific artifacts, facilitating collaboration and knowledge discovery.

By integrating these technologies with Open Science principles, researchers can create a robust and transparent environment for data sharing, collaboration, and knowledge discovery, ultimately leading to more reliable and reproducible findings.


\end{document}
